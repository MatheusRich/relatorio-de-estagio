\chapter[Introdução]{Introdução}

\section{Motivação}
Este estágio foi realizado à partir da Fundação de Empreendimentos Científicos e
Tecnológicos, desenvolvendo uma aplicação para a empresa CLAMPER. A CLAMPER
oferece ao mercado soluções para proteção de equipamentos eletroeletrônicos
contra danos causados por raios e surtos elétricos \cite{clamper}. Este
trabalho, em específico, lidou com dispositivos de prevenção de surtos (DPS),
com o objetivo de coletar informações sobre suas vidas úteis ao longo tempo.

DPSs são ferramentas capazes de detectar sobretensões transitórias na rede
elétrica e desviar as correntes de surto. Porém, ao longo de seu ciclo de
atuação, os aparelhos sofrem degradações. Assim, é possível traçar uma
correlação entre o número de eventos de surto e sobrecarga com a durabilidade do
DPS. O objetivo deste estágio foi, em parceria com alunos e professores de
diversas engenharias, traçar uma modelagem capaz de descrever esta relação, de
forma a possibilitar a visualização da vida útil dos dispositivos e, portanto,
realizar manutenções preventivas.

\section{Objetivos}
Os meus objetivos no estágio estão elencados nos tópicos abaixo:

\begin{itemize}
  \item Definição de protocolos, tecnologias e modelagem de solução para o
  cadastro, gerência e visualização de DPSs e seus eventos (surto ou sobrecarga);

  \item Implementação do \textit{back-end} e \textit{front-end} do sistema proposto;

  \item Implantação dos sistemas desenvolvidos;

  \item Atuar e auxiliar as outras equipes envolvidas no projeto.
  
\end{itemize}