\chapter[Considerações Finais]{Considerações Finais}

A experiência de estágio é fundamental para a formação de um profissional de
qualidade, pois alinha a teoria aprendida ao longo do curso à problemas reais do
mercado. O crescimento não só profissional mas pessoal obtido durante este
período é amplo, dado o contato com outros profissionais. Considero minha
vivência particularmente enriquecedora visto a interação com alunos de outras
engenharias e inclusive colaboradores de áreas externas à engenharia, como o
designer que atuou na prototipação da aplicação.

Como descrito nos tópicos anteriores, o processo envolveu diversas áreas da
Engenharia de Software, como a coleta de requisitos, metodologias de
desenvolvimentos, testes, controle de qualidade e a codificação propriamente
dita (nos âmbitos \textit{front-end}, \textit{back-end} e infraestrutura).

Em síntese, julgo imperativa a prática do estágio, pois solidifica os
fundamentos previamente obtidos, bem como angaria novos conhecimentos tanto em
áreas técnicas quanto interpessoais.
