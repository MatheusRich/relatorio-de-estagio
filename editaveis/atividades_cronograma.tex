\chapter[Atividades e Cronograma de Execução]{Atividades e Cronograma de Execução}
\section{Atividades Desenvolvidas}
O período de estágio envolveu as seguintes atividades listadas na Tabela \ref{atividades}.

\begin{table}[h]
  \caption{Atividades desenvolvidas durante o período de estágio}
  \label{atividades}
  \begin{tabular}{|l|l|}
    \hline
    \multicolumn{1}{|c|}{\textbf{ID}} & \multicolumn{1}{c|}{\textbf{Atividade}}                                                                                                                                                                           \\ \hline
    01                                & \begin{tabular}[c]{@{}l@{}}Estudo dos protocolos de comunicação, tipos de dados a serem coletados pelo serviço\\ de \textit{back-end} e definição de tecnologias a serem utilizadas\end{tabular} \\ \hline
    02                                & \begin{tabular}[c]{@{}l@{}}Estudo das informações a serem apresentadas para os usuários (\textit{front-end})\\ e definição de tecnologias a serem utilizadas\end{tabular}                        \\ \hline
    03                                & Desenvolvimento do serviço de \textit{back-end}                                                                                                                                                  \\ \hline
    04                                & Desenvolvimento do serviço de \textit{front-end};                                                                                                                                                \\ \hline
    05                                & \textit{Deploy} dos serviços desenvolvidos                                                                                                                                                       \\ \hline
  \end{tabular}
\end{table}

Segue uma descrição detalhada de cada uma etapas:
\begin{itemize}
  \item \textbf{Atividade 01:} Foram levantados requisitos em
  conjunto com os professores orientadores e os representantes da empresa.
  Optou-se pelo desenvolvimento de uma API web seguindo o protocolo MQTT para a
  comunicação com os microcontroladores. A aplicação foi desenvolvida com o
  \textit{framework} Django REST, em Python;

  \item \textbf{Atividade 02:} Com base nas necessidades do cliente, escolheu-se
  construir o (\textit{front-end}) com o framework Vue, em Javascript. A
  comunicação entre \textit{front-end} e \textit{back-end} ficou sob
  responsabilidade do protocolo REST HTTP;

  \item \textbf{Atividade 03:} O desenvolvimento foi realizado totalmente em
  conjunto com os alunos das outras engenharias, de forma a adequar o softwares
  desenvolvidos por ambas as partes. Além disso, foram desenvolvidos testes para
  garantir o funcionamento da aplicação em cada iteração;

  \item \textbf{Atividade 04:} O \textit{front-end} pode ser construído em
  poucas etapas dado o trabalho com o designer contratado. O \textit{framework}
  escolhido mostrou-se eficiente para tornar dinâmicos os protótipos previamente
  desenvolvidos pelo designer;

  \item \textbf{Atividade 05:} O \textit{deploy} também foi facilitado visto que
  já havia uma estrutura disponível no LAPPIS, bem como os scripts construídos
  com as ferramentas Docker/Docker Compose.
\end{itemize}

\section{Cronograma de Execução}
A Tabela \ref{cronograma} demonstra o período em que as atividades do estágio
foram realizadas. Nota-se que inicialmente ocorreram diversas atividades
relacionadas ao levantamento de requisitos da aplicação a ser desenvolvida. Com
base nas necessidades das partes interessadas foram escolhidas as tecnologias e
deu-se início à codificação do produto. O processo foi realizado em sua
integridade em parceria com alunos das engenharias Mecatrônica e Eletrônica. Por
fim, último período englobou a implantação das partes (\textit{front-end} e
\textit{back-end}) na infraestrutura localizada no LAPPIS.

\begin{table}[h]
  \caption{Cronograma de execução das atividades do estágio}
  \label{cronograma}
\centering
\begin{tabular}{|c|c|c|c|c|c|c|}
\hline
\textbf{Atividade/Mês} & \textbf{Abril} & \textbf{Maio} & \textbf{Junho} & \textbf{Julho} & \textbf{Agosto} & \textbf{Setembro} \\ \hline
01                     & X              & X             &                &                &                 &                   \\ \hline
02                     &                & X             & X              &                &                 &                   \\ \hline
03                     &                &               & X              & X              & X               & X                 \\ \hline
04                     &                &               &                & X              & X               & X                 \\ \hline
05                     &                &               &                &                &                 & X                 \\ \hline
\end{tabular}
\end{table}
