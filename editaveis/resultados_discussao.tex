\chapter[Resultados e Discussão]{Resultados e Discussão}
O estágio realizado tratou-se da idealização, construção e implantação de uma
aplicação no contexto de DPSs. A aplicação foi construída e implantada com
sucesso e muito disso se deve ao bom planejamento realizado em contato
constante com a empresa.

É notável a influência das prática ágeis adotadas no processo de
desenvolvimento: Não apenas a comunicação constante com o cliente foi
priorizada, mas o contato entre as partes internas do projeto. Isso se refletiu
nas diversas reuniões de planejamento e acompanhamento realizadas, bem como a
comunicação rápida permitida por grupos em redes sociais como Telegram. Essas
ações derivam-se de práticas conhecidas do Manifesto Ágil, como a valorização de
interações, colaboração com o cliente, entrega contínua e resposta à mudanças
\cite{beck2001agile}.

As ferramentas de desenvolvimento escolhidas mostraram-se eficientes. Foi possível produzir uma API REST em um pequeno período de tempo com Django Rest Framework. Vue.js mostrou-se eficiente ao dinamizar os protótipos HTML desenvolvidos pelo designer, sem perca alguma. Por fim a conteinerização dos serviços via Docker/Docker Compose não só tornou simples o gerenciamento de dependências em desenvolvimento, como a implantação em produção dos aplicativos.
